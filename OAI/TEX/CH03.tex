\chapter{买入看涨期权}
买入看涨期权策略的成功,主要取决于投资者是否能选到会上涨的股票,以及能把握住这个机会。
\section{为什么要买入看涨期权}
买入看涨期权的最大吸引力是它们能为投机者提供很大的杠杆。投资者有可能在标的股票价格只有小幅上涨的时候,仍能实现很大的盈利。此外,尽管盈利的百分比很大,风险却不会超过一个固定的金额,也就是最初买入看涨期权所支付的价格。看涨期权必须全额付清,它们没有保证金的价值,也不能充当保证金。请注意,上面所说的关于必须全额支付期权权利金的说法不一定适用于长期期权(LEAPS),LEAPS 在 1999 年定为可以用保证金交
易。

有的投资者买入看涨期权的目的是想在把风险控制在一个固定金额的情况下,为其组合增加某种上行潜在收益。例如,如果某个投资者在想限制下行风险时通常只购买低波动率、保守型的股票,那么,他也许可以考虑将小部分现金放到波动性更大的股票的看涨期权上。按照这种方式,他可以“交易”比他通常交易的股票的风险更高的股票。如果这些波动性较高的股票价格上升,投资者就会得到可观的盈利。而如果它们大幅度下跌,就像波动性这个词所隐含的那样,由于持有的是看涨期权而不是股票,这个投资者的以金额来衡量的风险也是有限的。

有的投资者是因为另外一个理由而买入看涨期权,就是能够在不错过市场的情况下按合理的价格买入股票。\textbf{直接指定止损的价位}

不想“错过市场”的投资者有时还会用另一种方法来买入看涨期权。\textbf{错配资金,不想错过上涨}
\section{买入看涨期权的风险和收益}
对看涨期权的买家来说,最重要的是要意识到,一般而言只有在股票价格上涨的时候他才有可能会盈利。由于没有在买入看涨期权时对各种可以使用的期权进行风险和收益分析,结果还是赔了钱。他选对了股票,可是买错了看涨期权。

因为看涨期权买家的最好同盟是标的股票的向上运动,因此标的股票的挑选就成了看涨期权买家要做的最重要选择。对买入看涨期权来说,抓住正确的时机非常重要,因此,在挑选股票时,技术分析也许比基本面因素更为重要。看涨期权买家还会有另一个同盟军,但这个同盟军通常是他预计不到的:如果他的看涨期权的标的股票价格变得波动很大,看涨期权的价格就会上涨,以反映这种变化。

相对于购买实值看涨期权,购买虚值看涨期权的潜在风险和潜在收益都更大。对看涨期权的买家来说,绝对的以金额来衡量的价格不应当是一个决定因素。如果标的股票价格有显著的上涨,虚值看涨期权自然能提供更大的收益。但是如果股票价格的上涨幅度不大,实值看涨期权的表现就会更好一些。

相对于虚值看涨期权,实值看涨期权无价值到期的可能性要小得多。对看涨期权的买家来说,剩余存续期的长短也有关系。如果股票价格与行权价相当接近,那么,近期看涨期权就会紧跟标的股票的价格运动,因此,它的收益最高,风险也最大。远期看涨期权,由于它距到期日还有很长时间,因而风险最小,百分比收益也最小。中期看涨期权的风险和收益都是中等的,因而通常最有买入的吸引力。
\subsection*{时机的确定}
如果投资者相当肯定标的股票马上就会上涨,他就应当努力得到更多的收益,而不必那么担心风险。这就意味着应买入短期的、稍稍虚值的看涨期权。当然,这只是一般的规则。无论在什么情况下,投资者一般都不要买入离到期只有一个星期的虚值看涨期权。另一方面,如果投资者在时机把握上非常欠缺,他就应当买入长期的看涨期权,这样,如果他对时机的把握差错很大,可以降低他的风险。
\subsection*{delta}
与中间区域相比,这些曲线在两头时与“内在价值”线的距离更近,这就意味着时间价值在股票价格等于行权价时最大,在股票价格偏离行权价时最小,无论是向实值方向还是向虚值方向偏离。
\figures{fig3-1}{3 个月、6 个月和 9 个月看涨期权的定价曲线}

买家应当熟悉看涨期权的另一个特性,就是期权的对冲比率(delta)。简单地说,期权的 delta 就是当标的股票运动1点时,看涨期权的价格会上涨或下跌的数量。因为人们通常根据标的股票1整点的变化来讨论期权价格的变化,(而不是依据价格中的“即时”变化),于是就产生了上行 delta(up delta)和下行 delta(down delta)的概念。在真正的数学意义上,只有一种 delta,它衡量的是“即时的”价格变动。
\section{买什么样的期权}
交易策略有不同的种类,有的是短期的,有的是长期的(有的甚至是买来放在那里的)。如果某个投资者决定要使用期权来实施一个交易策略,在买入什么样的期权方面,该策略的时间范围往往主宰了该期权的一般类别:实值的还是虚值的,近期的还是远期的,等等。一般的规则是:策略的期限越短,用来交易这
个策略的工具的 delta 就应当越高。
\subsection*{日内交易}
许多希望使用期权的日内交易者没有意识到,就日内交易而言,应当使用 delta 值最高的工具。这个工具就是 delta 为 1.0 的标的资产。在这种很短的交易周期内,期权的买卖价差很难驾驭。期权的买卖价差要比标的工具自身的买卖价差相对更宽。此外,日内交易者所要抓住的,只是标的物每日运动中的一小部分,而实值或虚值期权并不能有效地反映出这些运动来。也就是说,如果 delta 太低,期权的日内交易者就没有空间来盈利。
\subsection*{短线交易}
假定某个投资者想通过使用某个策略来达到持有标的物 1-2 个星期的目的,就像在日内交易的情况中一样,在这种情况里,他也需要较高的 delta 值。
\subsection*{中线交易}
随着投资者交易策略周期的拉长,使用较低 delta 值的期权就更为合适。这通常意味着在选择时机方面不需要再那么精确。
\subsection*{长线交易}
如果投资者的策略是个长期策略,他就应当考虑 delta 更低的期权。这样的策略在把握时机的能力方面一般都很模糊,例如,根据对某个公司基本面的泛泛了解而选择买入它的股票。极端情况下,它甚至可以用到“买入并持有”的策略中。
\section{高级选择标准}
因买入目的而对看涨期权排序时,选择标准应该基于标的股票的波动率。任何只考虑标
的股票的百分比变化而不考虑它们的波动率,并由此得到的期权排序,都是没有价值的,因而不能被采用。(不同波动率的标的在未来一段时间的涨跌幅是不一样的)

对看涨期权的买家来说,潜在收益的正确排序方法可以表达如下:
\begin{enumerate}
    \item 假定每只股票在一个固定的时段(30、60 或 90天)内都会根据其波动率而上涨;
    \item 对每次上涨后的看涨期权价格进行估计;
    \item 按照进攻型买入方式对所有潜在的买入看涨期权行为按最高百分比收益机会进行排序;
    \item 假定每只股票的价格都会跟随其波动率而下跌;
    \item 对每次下跌后的看涨期权价格进行估计;
    \item 按照收益/风险比(用第 5 步得出的亏损百分比去除第 2 步得出的收益百分比)对所有可购买的看涨期权排序。
\end{enumerate}

从第 3 步得出的名单会产生比较激进的购买方式,因为它只考虑了潜在收益。从第 6 步得出的名单的投机性会比较小一些。
\subsection*{被高估或被低估的看涨期权}
投资者在决定应当买入什么样的看涨期权时,不应当只是以这个看涨期权是否被低估为根据。\textbf{如果是套利可以考虑}
\subsection*{时间价值是一个误称}
即使是没有经验的期权交易者也必须懂得,期权的价格中不是内在价值的那部分,也就是我们通常叫做“时间价值”的那部分,实际上并不只是由时间价值构成的。的确,随着到期日的临近,时间最终会夺走期权价格中的这部分价值。但是,\textbf{当期权离到期日还有相当长的时间时,期权价值中更重要的一个组成部分实际上是波动率}。如果交易者认为它的标的股票会有相当大的波动率,那么这个期权就会很贵;如果他们的看法相反,这个期权就会相当便宜。
\section{后续行动}
在所持的看涨期权增值后取走部分盈利常常是个明智的行为。一般来说,让盈利滚动是成功交易的关键之一。
\subsection*{锁住盈利}
如果看涨期权持有者的运气足够好,标的股票相对迅速地上涨,那么他可以实施若干增强自己的头寸的策略。对于那些在应该拿走盈利还是应该继续持有头寸,以求在标的股票继续上升时获得更大的盈利,之间举棋不定的已经有未兑现盈利看涨期权持有者来说,这些策略是有帮助的。

某个投资者在标的股票是 48 时,以 3 点买入了一手 XYZ 10 月 50 看涨期权。股票随后涨到了 58。这个买家也许应当考虑将他的 10 月 50 看涨期权卖掉(这个期权现在可能值 9 点),又或者是采取其他一些行动,其中有的会涉及 10 月 60 看涨期权,它的售价是 3 点。在这个时候,这个看涨期权的买家可能会采取以下四种基本行动中的某一种:
\begin{enumerate}
    \item 卖出买入的看涨期权,将头寸平仓,从而拿走盈利;
    \item 卖出持有的 10 月 50 看涨期权,用部分收入买入 10 月 60 看涨期权;
    \item 构建一个价差头寸,持有 10 月 50 看涨期权的同时,再卖出 10 月 60 看涨期权;
    \item 什么都不做,继续持有 10 月 50 看涨期权。
\end{enumerate}

如果持有者选择卖出 10 月 50 看涨期权,他就有 6 点的盈利,再去掉手续费。不过这样的话,他的头寸就不复存在了,他不会再从这个看涨期权中得到好处,也不会损失任何已经得到的盈利。他将 6 点的盈利兑现了。这是上面四种方法中最不激进的一种。如果标的股票价格继续上涨,涨到 63 以上,其他3种策略的表现都会比将把全部看涨期权平仓的做法要好。但是,如果标的股票价格不是上涨而是下跌了,并在到期日时跌到了 50 以下,那么这个行动所带来的盈利是四种方法中最多的。

第四种方法也是一种简单的策略,它什么都不做。如果一直持有看涨期权至到期日,这个策略就是四种策略中风险最大的。

持有者卖掉当前持有的 10 月 50 看涨期权,然后用一部分收入买入次高行权价的看涨期权,这种策略叫作\textbf{向上挪仓}(roll up)。按这种方式“向上挪仓”的买家从本质上说是在用别人的钱进行投机。他把自己的钱放回了自己的口袋,使用积累的盈利来博取进一步的收益。

在继续持有 10 月 50 看涨期权的同时,卖出 10 月 60 看涨期权。这就构造了一个所谓的\textbf{牛市价差}(bull spread)。这个价差是没有风险的,因为该价差中买入的那条腿,也就是 10 月 50 看涨期权,花费的是 3 点,而价差中卖出的那条腿,也就是 10 月 60 看涨期权,通过卖出而收入了 3 点。另一方面,这个价差的最大潜在盈利是 10 点,也就是 50 与 60 这两个行权价之间的差。如果在到期日 XYZ 高于 60,这个最大潜在盈利就会实现。

\begin{figure}
    \centering

    \begin{tikzpicture}[scale=.4,shift={(50,0)}]
        % 轴
        \draw[->] (42, 0) -- (70, 0) node[anchor=north] {标的资产价格};
        \draw[->] (50, -4) -- (50, 17) node[anchor=east] {收益};
        \draw[-,red,thick] (45, -3) -- (50, -3) -- (70, 17);
        \draw[-,orange,thick] (45, 0) -- (50, 0) -- (60, 10)--(70, 10);
        \draw[-,blue,thick] (47, 0) -- (60, 0) --(68, 16);
        \draw[thick] (43, 6)  --(68, 6);
        \draw[dashed] (60, 0)  --(60, 10);
        \draw[dashed] (50, 10)  --(60, 10);
        \node[anchor=south west] at (50, 6) {6};
        \node[anchor=south west] at (50, 10) {10};
        \node[anchor=south west] at (50, -3) {-3};
        \node[anchor=north west] at (60, 0) {60};
        \node[anchor=north west] at (67, 14) {(67, 14)};
        \node[anchor=north west] at (63,6) {(63, 6)};
        \node[anchor=north west,text=orange] at (65,10) {bull spread};
        \node[anchor=north west,text=blue] at (62,4) {roll up};
        \node[anchor=north west] at (45,6) {close};
        \node[anchor=north west,text=red] at (55,2) {Hold};
        \fill (67, 14) circle (5pt) ;
        \fill (63, 6) circle (5pt);
        \fill (60, 10) circle (5pt);

        % \node at (3, 1) {盈利};
    \end{tikzpicture}
    \caption{锁住盈利到期日的盈利结果}
\end{figure}
\subsection*{防范行动}
当标的股票价格下跌时,看涨期权的买家有时会使用下面的两种策略。两者都涉及价差策略,也就是说,同时买入和卖出同一标的股票上的不同看涨期权。

“向下挪仓(Rolling Down)”。如果期权的持有者拥有一个期权,这个期权目前有未兑现的亏损时,这样做就有可能大大增加当股票价格有一个相对小的反弹时得到一笔有限盈利的机会。在某些情况下,投资者可以在不增加风险,或者只增加很小风险的情况下实施这个策略。

许多看涨期权的买家都经历过类似这样的情景:最初用 3 点买入 1 手 XYZ 10 月 35 期权,希望股票价格会很快上涨。结果股价下跌了,例如跌到了 32。随着 10 月到期日的临近,这个看涨期权现在只值 1.50。如果期权的买家仍然认为在到期日前股票会略有反弹,那他可以继续持有这个看涨期权,或者“向下均摊价”(average down)(按 1.50 买入更多的看涨期权)。此时,投资者持有 10 月 35 看涨期权。投资者可以卖出 2 手 10 月 35 看涨期权,与此同时,买入 1 手 10 月 30 看涨期权。如果不考虑手续费,这样做不需要额外的投资。实施向下挪仓策略的关键:投资者可以通过基本相等的钱买入 1 手行权价较低的看涨期权,同时卖出 2 手行权价较高的看涨期权。现在头寸变为:1 手 XYZ 10 月 30 看涨期权多头 1 手 XYZ 10 月 35 看涨期权空头。通过这个交易,显著地降低了自己的盈亏平衡点,却没有增加风险。不过,最大潜在盈利却因此而受到限制:如果标的股票大幅反弹,就不能在这个机会中大幅获利了。
\begin{figure}
    \centering
    \begin{tikzpicture}[scale=.7,shift={(35,0)}]
        \draw[->] (27, 0) -- (42, 0) node[anchor=north] {标的资产价格};
        \draw[->] (35, -4) -- (35, 4) node[anchor=east] {到期盈亏};
        \draw[thick] (28,-3)--(35, -3)--(41, 3);
        \draw[thick,orange] (29,-3)--(30, -3)--(35, 2)--(41, 2);
        \draw[dashed] (30, 0)node[anchor=north west] {30}--(30, -3) node[anchor=north west] {-3};
        \draw[dashed] (40, 0)node[anchor=north east] {40}--(40, 2)node[anchor=north west] {2};
        \fill (30, -3) circle (3pt);
        \fill (38, 0) circle (3pt);
        \fill (35, 2) circle (3pt) node[anchor=north west,text=orange] {roll down};
        \node[anchor=north west] at (36, -2){最初的买入看涨期权};
    \end{tikzpicture}
    \caption{比较:最初的买入看涨期权与价差}
\end{figure}

只需要略有反弹就可以实现盈利,而不需要从 32 上涨到 38 以上才能盈利。

这个示例特别有吸引力,因为建立这个价差不需要额外的资金。不过,在许多情况下,投资者会发现把买入的看涨期权变成价差时的收支并不相等。会出现一些支出。这一事实并不是说不应当改变策略,因为只需要小部分的额外投资,就可以显著地增加在股票反弹时达到盈亏平衡甚至得到盈利的机会。

建立跨期价差(calendar spread)。当看涨期权的买家发现标的股票价格下跌时,有时他们可以采用另一种防御型的价差策略。在这种策略中,某个中期或远期看涨期权的持有者可以卖出一手相同行权价,但存续期更短的看涨期权。这就构造出了所谓的跨期价差。这样做的理由是,如果近期的看涨期权无价值到期,那这个买家买入看涨期权的总成本就降低了。如果股票价格上涨,这个看涨期权买家的盈利机会就增加。

在使用这个策略时要格外小心。因为如果在近期期权到期之前标的股票迅速上涨,这个价差就会在两条腿上都有亏损。