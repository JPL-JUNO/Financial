\chapter{其他买入看涨期权的策略}
我们将讨论另外两种买入看涨期权的策略。这两种策略都涉及卖空标的股票和买入看涨期权。当标的股票有场内看跌期权交易的时候,这些策略就没有使用看跌期权那么好。不过,这里的概念相当重要,并且当市场中看涨期权非常活跃而看跌期权不活跃时,这些策略就会更有活力。这些策略一般被称为“合成”(synthetic)策略。
\section{保护性卖空(合成看跌期权)}
在卖空标的股票的同时买入看涨期权,是将卖空的风险限制在一定数额里的一种手段。从理论上来说,卖空的风险是无限的,因此许多投资者在卖空时都会有所顾虑。对这些卖空股票的投资者来说,股票价格的上涨会让他们心绪不安。投资者有可能会由于情绪的缘故而被迫作出也许是不正确的决定——回补卖空头寸,以减低心理压力。如果在卖空股票的同时持有看涨期权,投资者就可以把亏损限制在一个固定的、通常是相当小的数额内。

当投资者买入看涨期权来保护卖空头寸时,有一个简单的公式可以计算最大风险金额:
\begin{equation}
    \text{风险}=\text{买入的看涨期权的行权价}+\text{看涨期权价格}–\text{股票价格}
\end{equation}

无论是上涨还是下跌,卖空者的风险都会因标的股票发放股息而略有增长,因为他必须为卖空的股票支付股息。

一般而言,最好是买入平值或略微虚值的看涨期权来对卖空头寸进行保护。买入深度虚值看涨期权在保护方面起不到什么作用,除非股价急剧上涨,否则它对风险没有什么改善。正常情况下,投资者会在卖空头寸对其产生严重不利后果之前就回补。因此,花钱买入这样一个深度虚值看涨期权是一种浪费。不过,如果投资者想要他的卖空头寸有足够的“活动”空间,并且非常肯定他对这个股票极度看空的看法是正确的,那么他可以买入相当深度的虚值看涨期权来作为灾难保护,以防股票价格突然向上爆发(例如,标的股票突然收到收购要约)。

\begin{figure}
    \centering
    \begin{tikzpicture}[scale=.4,shift={(35,0)}]
        \draw[->] (35, 0) -- (47, 0) node[anchor=north] {\small{到期时标的资产价格}};
        \draw[->] (35, -6) -- (35, 6) node[anchor=east] {\small{到期时盈亏}};
        \draw[thick] (35,2)--(37,0)node[anchor=north east]{0}--(40,-3)--(45, -3);
        \draw[dashed] (37,3)--(40,0)node[anchor=north east]{40}--(45, -5);
        \draw[dashed] (40, 0)--(40, -3)node[anchor=north east]{-3};
    \end{tikzpicture}
    \begin{tikzpicture}[scale=.4,shift={(35,0)}]
        \draw[->] (35, 0) -- (47, 0) node[anchor=south] {\small{到期时标的资产价格}};
        \draw[->] (35, -6) -- (35, 6) node[anchor=east] {\small{到期时盈亏}};
        \draw[thick] (36,3.5)--(39.5,0)node[anchor=north east]{0}--(45, -5.5)--(46,-5.5);
        \draw[dashed] (37,3)--(40,0)node[anchor=south west]{40}--(46, -6);
        \draw[dashed] (45, 0)--(45, -5.5)node[anchor=north east]{-5.5};
        \fill (45, 0) circle (3pt);
    \end{tikzpicture}
    \caption{卖空者愿意承担的风险可能不同,他也许想买入1手虚值看涨期权作为保护,而不是上面示例中的平值看涨期权。如果买入的是虚值看涨期权,保护成本就会低一些,卖空者所放弃的潜在盈利也少一些。但是他的风险就会大一些,因为只有在股票上涨到行权价之上的时候,这个看涨期权才具有保护功能。}
\end{figure}
\section{保证金要求}
根据最新的保证金规则,如果股票空头头寸有看涨期权多头保护,投资者在保证金要求方面就会有相当的优待。实际所需的保证金为下列两项的较小值:第一,看涨期权行权价的 10\% 加上虚值部分的金额;第二,卖空股票现有市场价值的 30\%。这个头寸会被逐日盯市,如果股票价格低于行权价,大部分经纪商会要求该卖空头寸按“正常”比率缴纳保证金。
\section{组合保证金}
一般而言,组合保证金要求是基于风险的,很难手工计算出来。
\section{后续行动}
保护性卖空者在这个策略中需要采用的后续行动基本上就是平仓。如果标的股票先迅速下跌,然后看上去会反弹,那投资者应该回补股票,而不是卖出看涨期权。这样做的话,如果股票反弹到最初的行权价之上,投资者还能从看涨期权中获利。如果标的股票价格上涨,那就不应该采取相似的、只卖出盈利头寸(看涨期权)的方法。也就是说,如果 XYZ 从 40 涨到了 50,而 7 月 40 看涨期权价格也从 3 涨到了 10,那就不应只卖出看涨期权获得 7 点盈利,并继续持有股票以希望其会下跌。理由是,当看涨期权为实值时,如果解除保护,这个投资者就会面临高度的风险。如果股票价格下跌,那么提走盈利就不是问题,这甚至是他所期望的。因为如果股票继续下跌,就没有或者只有很小的额外风险。但当股票上涨时,情况就不同了。在这种情况下,如果卖空者卖掉他的看涨期权拿走盈利,而股票随后继续上涨的话,就会产生大笔的亏损。

如果看涨期权是持平(at parity)或接近持平的,或者是实值的,那么通过行权而把头寸平仓,就常常是可取的做法。在大多数策略里,由于股票手续费比期权手续费高出许多,将看涨期权行权对期权持有者来说没有什么好处。但在保护性卖空的策略里,卖空者最终总是要回补他卖空的股票,因而总会有股票手续费。因此,行权并按行权价(也就是较低的价格)买入股票,或许还能因此少支付些手续费,这也许会给他带来好处。
\section{合成跨式价差(反向对冲)}
在这种策略里,投资者买入的看涨期权所对应的股数要多于其卖空的股数。如果在期权的存续期内标的股票上涨或下跌的幅度足够大,这个策略家就可以盈利。这个策略一般被称作反向对冲(reverse hedge)或合成跨式价差(synthetic straddle)。如果该股票有场内看跌期权交易,那这个策略就过时了,直接买入跨式价差(1 手看涨期权和 1 手看跌期权)所产生的结果会更好。因此,这个反向对冲策略又被称为“合成跨式价差”。

如果股票在任何方向上有足够幅度的运动,显然都可以得到盈利。事实上,投资者可以准确地判断出,如果要盈利,在到期时股票必须达到怎么样的价格。这些盈亏平衡点很容易计算出来。首先计算出最大风险,然后再确定盈亏平衡点。
\begin{equation}
    \begin{aligned}
        \text{最大风险}    & =\text{行权价}+2\times \text{涨期权价格}-\text{股票价格} \\
        \text{上行盈亏平衡点} & =\text{行权价}+\text{最大风险}                      \\
        \text{下行盈亏平衡点} & =\text{行权价}-\text{最大风险}                      \\
    \end{aligned}
\end{equation}

在到期之前,即使股价离行权价很近也有可能盈利,因为买入的看涨期权还剩有时间价值。

一般而言,进行合成跨式价差交易的股票的波动率应该较大。尽管这种股票的期权权利金会比较高,但当价格呈直线运动时,股票价格的变化幅度仍会大于这些权利金。使用波动率较大的股票的另一个好处是,一般它们很少或者没有股息。这是合成跨式价差所期望的,卖空者也就不必付或者只需付很少的股息。

在建立头寸时,标的股票的技术形态也会有帮助。交易者一般希望在策略亏损的区域内不存在技术性的支撑位和压力(resistance)位。在这样的形态里,股票可以上下快速运动。交易者有时也可以交易宽幅震荡的股票,它们的股价不断地在震荡区域的这一端摆动到另一端。如果反向对冲的亏损区域刚好位于该震荡区间之内,那这个头寸也会有吸引力。
\section{后续行动}
因为反向对冲内在的有限亏损特征,所以没有必要采取任何后续行动来限制亏损。投资者可以相当容易地建立头寸,在到期日前也不必采取任何后续行动。在这个策略里,这经常就是最好的后续行动。

另外,还有一种后续行动可以运用,尽管这种行动有一定的不利之处。它有时被称为针对跨式价差的交易(trading against the straddle)。当股价在任何一个方向运动得足够远的时候,就从这一侧提取盈利。然后等股价摆回到另一个方向时,就再从另一侧提取盈利。
\section{改变看涨期权多头和股票空头之间的比率}
资者不一定非要刚好买入 2 手看涨期权来对应 100 股股票空头。他可以就 100 股股票空头买入 3 或 4 手看涨期权,以建立一个更为看多的头寸。

不管比率是多少,都可以用一个公式来计算最大风险和盈亏平衡点。
\begin{equation}
    \begin{aligned}
        \text{最大风险}    & =(\text{行权价}-\text{股票价格})\times \text{卖空的股票手数}+\text{买入的看涨期权手数}\times \text{看涨期权价格} \\
        \text{上行盈亏平衡点} & =\text{行权价}+\frac{\text{最大风险}}{\text{买入的看涨期权手数}-\text{卖空的股票手数}}                     \\
        \text{下行盈亏平衡点} & =\text{行权价}-\frac{\text{最大风险}}{\text{卖空的股票手数}}                                      \\
    \end{aligned}
\end{equation}

这个策略可以使用的最后一种调整方法,就是在卖空 100 股股票的同时,买入 2 手行权价不同的看涨期权。于这个策略涉及两个行权价,因此被称为“合成宽跨式”(synthetic strangle),即由两个行权价不同的看涨期权或看跌期权构成的普通跨式。
\section{总结}
如果标的股票有场内看跌期权,这一章所描述的策略一般就没有用处。但是,如果没有看跌期权存在,或者看跌期权很不活跃,而策略家认为在看涨期权的存续期内股票会在某一方向上有相对较大的运动,他就应当考虑使用某种形式的合成跨式策略,也就是卖空一定数量的股票,同时买入对应更多股票数量的看涨期权。如果他所希望的运动确实出现了,就会产生可观的盈利。无论是哪种情况,亏损都被限制在某个固定的金额之内,一般是初始头寸的 20\%-30\%。虽然可以采取一些后续行动来锁住小额盈利和从股票的反向运动中获利,但更明智的做法是继续持有头寸,以便获得更大的盈利。这个策略一般使用 2:1 的比率(买入 2 手看涨期权,卖空 100 股股票),但如果投资者想要更为看多或者看空,可以调整这个比率。如果初始股票价格在两个行权价之间,可以通过在卖空股票的同时分别买入次高行权价和次低行权价的看涨期权,来建立一个中性的盈利区域。