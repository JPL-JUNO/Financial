\chapter{卖出裸看涨期权}
\section{所需投资}
\paragraph{裸期权头寸是要逐日盯市的。}这就意味着这个头寸的质押要求与卖空股票一样,是每天重新计算的。需要进行逐日盯市的理由很明显。如果标的股票价格上涨,经纪公司就必须能够保证,如果这个裸看涨期权收到指派通知,客户有足够的质押物来在公开市场上买入股票,并按行权价卖掉股票。如果股票价格下跌,逐日盯市就对客户有利。多余的质押物就会转回到客户的保证金账户,并可以用来做其他交易。

对许多投资者来说,使用质押来卖出裸看涨期权很有吸引力,因为投资者不必改变他现有的投资组合就可以卖出看涨期权和收到权利金。当然,如果裸期权的标的股票价格上涨太多,由于逐日盯市的缘故,经纪商有可能会要求该投资者追加额外的质押物。此外,不管是用质押还是现金,都有风险存在。当投资者买回亏损的裸看涨期权时,他必须付出现金,在其中账户中产生一笔支出。
\section{卖出裸期权的哲学}
在考虑卖出裸期权(或者说,任何策略)的时候,投资者首先最需要问的问题是:“我在心理上能不能接受账户中的裸期权头寸?”

一般而言,如果可能的话,投资者应当让裸期权无价值到期,而不是去动它们,除非标的工具发生了大幅度的反向运动。因此,卖出裸期权一般是选择虚值期权。为了减少(或者近似取消)收到保证金追加通知的机会,投资者应当按照标的物已经运动到所卖期权的行权价这样的设想来预留保证金。由于已经预留了即使标的物价格等于行权价也可以应付的保证金,在标的物价格运动到行权价之前,该投资者基本上不会收到保证金追加通知。而一旦标的物价格达到了行权价,那最好把头寸平仓,或者是把这个看涨期权挪仓到另一个行权价上去。

跳空在股票中相当普遍,在期货中就很少见到,而在指数中则几乎不存在。因此,在卖出裸看涨期权时,应该选择指数期权。

卖出裸期权还必须遵守另一条“规则”:必须有人始终监控这个头寸。

总的来说,在卖出裸期权时,投资者必须在心理上做好准备,有足够的资金,愿意接受风险,能够每天监控他的头寸。他应当卖出隐含波动率极高的期权,并在期权变为实值时回补这个头寸。
\section{风险和收益}
投资者可以通过选择卖出实值或虚值裸看涨期权,来调整他的风险和收益。卖出虚值裸看涨期权,特别是深度虚值看涨期权,能够以很高的概率获得一笔小额盈利。而卖出实值裸看涨期权则有更大的潜在盈利,不过相应的风险也更大。

卖出虚值期权做既有有限的额外收入,又有很高的成功概率,因此许多投资者都被吸引过来,利用现有投资组合的质押价值,来卖出深度虚值裸看涨期权。

在使用这个技术时,一种有利的做法是在建立头寸时选择一个价格是或者接近 15 的股票,然后卖出行权价为 20 的近期裸期权。这个期权的价格或许是 1/8 或是 1/4,不过有的时候根本没有人在这个价位上提供买入报价。

卖出裸看涨期权的另一端是卖出深度实值的看涨期权。因为实值看涨期权没有多少时间价值,卖出者在上行方面就没有多大的活动余地。如果股票真的上涨,一般而言,深度实值看涨期权的裸卖出者会遭受亏损。但如果股票价格下跌,实值看涨期权的裸卖出者的盈利会大于虚值看涨期权的裸卖出者。卖出深度实值看涨期权的盈利与卖空股票的盈利相仿,至少在股票跌到接近于行权价之前是这样,因为深度实值看涨期权的 delta 接近于 1。

如果交易者希望卖空股票以抓住几点的价格变动,那他可以卖出深度实值裸看涨期权,而不用卖空股票。这样做的话,他的投资会比较小(卖出裸期权需要的投资是股票价格的 20\%,卖空股票则是 50\%),而收益则比较大(实值部分的保证金要求被收到的权利金抵消了)。卖出者需要对看涨期权的时间价值特别注意。他不想因为卖出看涨期权而收到指派通知。更长期的期权系列更容易有时间价值。\textbf{如果交易者想要从看空的角度交易一只股票,一般而言,卖出最远期的深度实值看涨期权是避免指派的最安全方法。}
\section{后续行动}
由于卖出裸看涨期权在理论上有很大的上行风险,投资者需要持续地监控他的头寸。限制亏损的最简单方法就是在心中设置某个止损价。

另一种后续行动就是,当没有挣更多钱的机会时,平掉头寸拿走盈利。投资者可以计算出继续持有该裸看涨期权头寸的剩余收益。如果该剩余收益太小,那就应该买回看涨期权,然后寻找其他交易机会。
\subsection{收入挪仓}
虽然这个策略对某些期权交易者有吸引力,但有一点需要避免,那就是卖出平值看涨期权。这对先前卖出的裸看涨期权所采取的后续行动也同样适用。此时股票价格上涨到行权价,以及在那一点卖出平值看涨期权。

在收入挪仓的策略中,投资者需要在股票价格达到另一个行权价时,平掉其看涨期权空头,并同时卖出更多的另一个行权价的看涨期权,同一个月份或更远一些的月份均可。他应该卖出足够的新看涨期权,以便让卖出较高行权价看涨期权所收到的权利金等于买回较低行权价看涨期权所支付的金额。投资者可以重复这样做,直到股票最终下跌,而让最后卖出的看涨期权无价值到期。到时他的盈利额为最初的收入,加上后续挪仓中可能得到的任何收入。不过在大多数情况下,挪仓都会需要少量的支出,因此盈利主要还是最初的收入。

当投资者月复一月地不断向上挪仓,则可能在风险和需要的质押方面存在问题,这会侵蚀最初的有限收入。

在这个策略中,股票的任何快速上涨都会带来问题。此外,如果股票暴涨并继续暴涨(比如收到收购要约,有大订单或建立生意伙伴关系等消息宣布)则会带来毁灭性的打击。

这实际是一个“马丁格尔策略”(Martingale strategy)。如果看涨期权是备兑的,这类策略会更有效一些(参考卖出备兑中的“增额收益概念”)。这些分析对于卖出裸看跌期权也同样适用。在这种情况下,股票价格不会下跌至零,但也会因质押要求变得非常大而难以维持头寸。
\subsection{时间价值是一个误称}
把期权价格中不是其内在价值的那部分称作“时间价值”,这并没有什么不对,不过有见识的期权交易者知道,波动率和股票价格的运动对这部分价值的影响,要比时间流逝对其的影响大得多。
\section{总结}
大多数情况下,卖出裸看涨期权都作为一种深度虚值期权的策略来使用,其中,投资者用所持证券的质押价值,来参与一种能以较高概率获得非常有限的收益的策略。这是一个不高明的策略,因为一次亏损可以抹掉许多次盈利。如果投资者想要寻找一种卖空股票的替代策略,那他可以卖出实值裸看涨期权,以求用一笔小于卖空股票所需的投资来谋取快速盈利。如果投资者没有足够的资本作后盾,这两种策略都可能会带来大量的风险。

为避免亏损变得不可收拾,采取后续行动很有必要。一般而言,当标的股票价格上涨至盈亏平衡点时,应该买回看涨期权。此时可以就另一个股票或另一个期权系列建立更好的头寸。