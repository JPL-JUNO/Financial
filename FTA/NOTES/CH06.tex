\chapter{期货实战之本色交易:K 线}
\section{K 线的几种走势结构}
我们的技术都在谈市场的结构,市场结构就是市场群体行为的法则,是会不断重复出现的模式。在结构形成的比较微观的层面,四个价格加上不同时间周期形成了不同周期的 K 线。基本趋势的运行是由一系列 K 线积累而成的。
\subsection{整理结构}
整理结构说明目前市场进入了横盘状态。

按照经验,整理结构基本划分为三种时间级别,快速的整顿通常时间比较短,4-6 天的 K 线整理足以达到目的。在一波强烈行情中,多方或者空方都希望速战速决,再加上外部市场环境的配合,强烈趋势的发动通常只需要短暂整理,有时甚至无须整理,直接一步到位。我们通常会错过加入趋势的机会,因为心态犹豫而错失了短暂的加仓机会。

如果价格进一步出现停顿,通常K线便扩展到 10-13 天。这时候出现了一个明显的多空平衡运行的区域,价格可能运行出现一种形态结构,然后继续沿着趋势方向出现新的突破走势。这样的行情给了我们充分加入趋势的时间和机会,但同时也容易造成趋势反转的假象,使很多喜欢提前摸顶抄底的交易者容易犯下错误。

如果 K 线继续运行,超过这个阶段还没有形成新的突破,则行情震荡会延续到 15-20 天,有时候甚至达到 30 个 K 线的时间周期。长期震荡不是一个好现象,多空双方难分胜负,形成长期胶着态势,而且很容易形成无序的震荡行情,也不太容易出现明显、可辨别的形态,操作难度加大。

再引申一下,整理一般价格波动偏小,而行情启动后趋势中价格波动通常很剧烈,持续的小波动后面必然会出现价格的大波动;同样,大起大落过后,价格再次进入小波动局面。这个现象也是交替出现的。所以,当行情进入整理阶段时,我们千万不要觉得无趣,这时恰恰是更需要关注行情、培养你的耐心的时候。
\subsection{反转结构}
我们通常定义的 M 顶部和 W 底部是标准的反转形态,也就是说,在 K 线走势上出现明显的 1-2-3 点,我们就可以先确认一个反转出现的迹象,如果 3 被向上或者向下突破,那么就可以在 K 线上完全定义一个反转结构的出现。

如果是底部,则先追踪寻找一个最低点,通常行情走出来后 1 点比较好确认。价格快速反弹上冲,回落后形成了一个高点,即 2 点。价格继续回落后没有再打破 1 点,稳定后又开始反弹,出现了一个低点,即 3 点。这个时候基本的 1-2-3 结构出现了。如果 1-2-3 结构形成后没有突破,或者出现假突破,那么可以判断行情继续进入整理阶段,但不能认为这个底部形态失败了,除非价格反向突破运行。随着价格在整理区间内运行,很可能再次突破。也就是说,1-2-3 结构的初步形成给了我们一个信号,即行情可能进入一个转折阶段,前面的趋势告一段落了。1-2-3 结构可以在任何阶段出现,如果出现在行情相对低位,那么可以认为这是反转的信号。

同样,如果是顶部,则价格往往会出现一个极值高点,之后价格迅速回落,出现 2 点,然后价格再次上攻,但这时候没有再创新高,3 点便形成了。同前面的价格线入场法则一样,K 线结构越清晰越好。这里有一种特殊情况需要说明,头部价格崩溃时容易出现 V 形反转形态,这时我们可能只能明显看到 1 点,所以 1 点是最好确认的,而下面的 2 点和 3 点会距离很远。
\subsection{延续结构}
从 K 线上可以简单定义,只要价格不断创出了一个新的高点或者低点,我们就可以理解为行情在延续。这里的高点我们定义为 K 线上的当日最高价格,而不是收盘价格创新高。

这里体现了对市场的理解,价格只要盘中创出了新高,虽然有可能收盘大幅度下跌,但行情毕竟还是出现了更高的点,这代表了资金推动趋势的力量,就好比价格突破了区间,但又回落形成了假突破,但突破毕竟还是发生了。如果价格创造了新高,但后面没有在一定时间内出现更高价格,那么很可能会形成一个K线的反转结构。所以,只要价格出现了新高或者新低,我们就定义为延续 K 线结构。
\begin{tcolorbox}
    有点理解为有上影线的存在,那么市场尝试冲高,如果是在趋势中,是可以保留的。
\end{tcolorbox}
\subsection{K 线经典组合}
\figures{fig6-3}{这两种 K 线形态比较有价值的是,它们通常出现在趋势运行的过程中,比如在下跌趋势中出现一个阳包阴,会让很多空单离场,对于追空的交易者一般会触发止损。阴包阳对多头上行的杀伤力也很大,常见洗盘模式。我们通常可以关注  1-2 天,如果行情没有继续下跌,就会进入整理形态,很可能趋势继续发展。}

十字星为最经典的 K 线形态,但经典不一定代表实用、好用,我们的经验告诉我们,单根十字 K 线通常不可靠,最好配合周边的 K 线来判断。当然,如果频繁出现,则一定有重要意义。切记不要因为看到十字星就立刻开出多单或者空单,往往第二天就算真的反转,也经常伴随着洗盘的出现。
\begin{tcolorbox}
    2024-09-30 恒生指数见到一个十字星,这让我直接清仓了恒生 ETF 和恒生科技,有点遗憾。
\end{tcolorbox}

\section{K 线实战案例}
\figures{fig6-9}{走出了很多种类型的 K 线结构,之前高位是典型的 V 形反转结构,在行情顶部比底部更容易发生这样的变化,没有经过长期,甚至是短期的整理过程就直接持续下跌,不断创新低。根据市场结构模式可以知道,行情进入震荡的概率比较大,这时我们应该关注时间因素。如果行情不能直接反转下跌,那么可能会形成一个超过 20 个交易日的长时间整理结构。}

注意,我们是从第 1 根不创新高的 K 线之后算起的,只要行情不延续,我们就假设它会整理,整理几天我们无法提前预知,但可以观察。

\figures{fig6-12}{该段行情的整理区间超过了 20 个交易日,但在区间内部我们要考察是否有明显清晰的反转 1-2-3 结构出现,我们找到了这样的 3 个点位。3 点之后几天内行情顺势突破,并刚突破后就形成了一个短暂强烈的整理结构,随后 5-6 个交易日左右,行情继续走出延续结构。我们经常遇到的市场中的各种走势行情,往往是不太规则的结构,但如果我们先把大的市场结构划分出来,那么我们就可以只利用 K 线找到好的交易位置。注意在 3 点位置,MACD 虽然有短暂的变绿,但是很亏就被上涨恢复成红色,这里的 MACD 与更高的低点吻合住了,验证了 MACD 绵延走势的趋势延续。}

\figures{fig6-22}{IF 股指期货日线走势图}

\autoref{fig6-22} 中,股指行情持续反弹,我们看到上行趋势不算稳定,中间过程波动很大,但依然走出了四段明显的整理结构。前两段短暂整理后,很快创了新高;后面两段整理时间相对比较长,但行情依然在延续。目前行情又进入了整理阶段,后面突破依然是我们的机会。