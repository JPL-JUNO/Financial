\chapter{投资组合管理}
\section{多品种,多策略,多周期组合}
投资组合也不是万能的策略,更不是无风险的绝招,股票中的投资组合最怕的是大盘持续下跌的系统性风险,这个时候组合中的所有股票基本都会下跌,只是跌多跌少而已,很难有盈利的机会。而期货市场当然也有系统性风险,比如2008年的经济危机,2008年国庆节过后,沪铜领跌,出现了连续不断跌停的行情,同时带动其他化工品下跌,甚至连农产品也难逃下跌的厄运。

合理的仓位资金管理是应对系统性风险的不二选择。如果当下你的投资组合表现不佳,其中也没有品种具有明显的盈利潜质,那么就说明市场整体氛围都不太好,应该适当减少组合的仓位,或者去掉一些行情走势很差的品种。根据行情的发展,投资组合可以适当集中仓位在好的品种上。

同时,投资组合的盈利性体现在不同品种的搭配上,这样可以捕捉到更多的潜在趋势性机会。东方不亮西方亮,行情总是会有的,但就某一个品种来说,一年之中的行情是有限的。因此,如果你的投资组合中选择的品种在一定时间阶段内没有体现出明显的盈利,那么这本身也会给账户整体带来风险。