\chapter{期货实战技术核心:市场结构交易}

首先,跟踪市场已经走出来的高点、低点。注意,只要在行情运行中走出来明显的高低点,那么这些点很可能就是市场不同位置的拐点,也就是反转点。大级别有大的反转,小级别有小的反转,总之,这个位置很重要。可惜的是,你几乎不可能抓到这个点。那么我们就以这个点为依据,找相对的高点和低点,然后再找之前的相对次高点和次低点,从而更好地实现低买和高卖。

其次,我们希望找到一个相对稳定的价格入市,就好比我们要上船一样,希望海面平静,不至于一上去就被浪打翻。而市场在形成最高点和最低点的时候通常是不稳定的状态,多空博弈很激烈。最低点往往是空头用尽最后的力量打压,但发现多头资金全部接住并且开始继续买入推高价格,这个时候空头力量逐渐消弱,行情由空转多,并且形成了一个最低的点位。在这之后市场会恢复相对平衡,需要多空双方重新积累力量才可以发动新的行情。

最后,我们就在这个最高点和最低点前面找一个比较明显的低点,如果发现这个位置还不够稳定,就再找一个点,这个位置就是我们要操作的入场价格。

我们对价格线入场点做一个总结:

\begin{enumerate}
    \item 找到明显的高点和低点是首要的参考标准。如果考虑做多,那么就找已经形成的最近的明显低点;如果考虑做空,那么就找已经形成的最近的明显高点。
    \item 图中位置越标准越好,也就是 1、2、3 点越清晰越好,它们之间位置空间距离适中比较合适,离得既不过近又不过远。如果1、2、3 点互相之间离得很近,那么可以合并成一个位置,再继续往前找更突出的位置。
    \item 有时候没有发现合适的 3 个位置,或者只找到了其中的 2 个点,那么可以参考操作入场或者观望放弃机会。
\end{enumerate}

当交易中出现连续止损时适当停止交易,等待区间形态明确并出现了更明确的位置后再进行交易。


在震荡行情阶段,短周期行情可能会出现转头的迹象,长周期行情波动显得很平稳,通过长周期可以对短周期的迹象进行过滤,有利于我们更好地持仓。

\begin{table}
    \centering
    \caption{期货品种交易复盘表}
    \begin{tabular}{lllllllll}
        \hline
        复盘时间      & 品种  & 5分钟 & 15分钟 & 60分钟 & 日 & 周 & 月 & 趋势评级 \\
        \hline
        2024-9-27 & 螺纹钢 &     &      &      &   &   &   &      \\
        2024-9-27 & 玉米  &     &      &      &   &   &   &      \\
        \hline
    \end{tabular}
\end{table}
\section{趋势工具实战应用}

\subsection{布林线实战应用}
\begin{table}
    \centering
    \begin{tabular}{lll}
        \hline
             & 上下轨表现     & 形成原因        \\
        \hline
        分开   & 上轨向上、下轨向下 & 价格产生趋势行方向变化 \\
        收敛   & 上、下轨向中轨收敛 & 价格逐渐进入横盘阶段  \\
        三轨同向 & 三轨同趋势方向运行 & 趋势持续运行中     \\
        整体走平 & 三轨横向运行    & 价格明显进入横盘阶段  \\
        \hline
    \end{tabular}
\end{table}

对布林线的应用提炼出了一些实用的内容,具体如下。

\begin{enumerate}
    \item 价格发动趋势变化前,很难站稳上、下轨线,一般都在上、下轨区间内震荡运行。
    \item 上行强势趋势和下行疲弱走势确认后,基本价格可以沿着布林线上、下轨运行一段时间,并且通常回调和反弹都到不了中轨,价格趋势沿着方向继续运行。
    \item 震荡区间价格也可能触及上、下轨,但会很快演变成支撑和阻力。震荡区间价格一般都会穿越中轨。
\end{enumerate}
\subsection{价格趋势线实战应用}

因为趋势线比较简洁,跟踪行情没有那么敏感,所以我们完全可以放大周期操作周线图。周线交易需要放慢心态,需要更多的耐心。