\chapter{期货形态实战应用}
\section{市场结构特征:价格形态实战应用}
技术分析并不热衷于提前预测,所以我们会根据价格走势信号先确认一些形态形成的迹象,然后判断出一个市场结构的框架。如果后面价格在这个框架中运行,那么这段走势肯定是比较顺畅和规则的,否则会根据价格走势及时调整对结构的判断。

最好的形态可以称之为积累突破式形态。通常接近于三角形,但价格走势收敛越窄越好。第二种是矩形震荡形态,这样的形态或者直接可以交易,或者可以让我们清晰地看出市场结构,便于分析,我们可以称之为稳定震荡形态。还有一种归纳为顺势跟进形态模式,一般见于不规则形态,走势不是很明确,但会有一点产生突破或者在某一个时刻出现方向的倾向。如果不想在形态内过多试错,等待出现明显突破后再择机顺势开仓跟进。我们称之为顺势加码形态。
\section{市场结构中三个重要位置的形态}
\subsection{震荡筑底形态}
首先,我们从趋势角度看,一旦趋势线被有效打破,则行情大概率进入震荡格局。如果我们并不操作震荡区间,那么接下来就要很明确地判断行情是停顿后维持原有趋势继续运行还是等待反转开启新的方向。我们之前的理论已经对市场结构走势做了清晰的判断和划分,形态结合趋势的操作主要有两个时机点:\textbf{一是形态突破的机会,往往伴随着有效的量能。二是突破后回踩进入机会。}

\figures{fig4-10}{我们很清晰地画出了 W 底部的三个点,要求第三点一定要高于第一点,也就是价格不再创新低了,前面的下降趋势节奏告一段落。价格积累运行后突破颈线,即图中下面的横线。这往往预示着底部的形成,形态最后确认都是以突破为依据的,但这并不保险,因为形态经常走坏,价格很可能突破失败回到区间内继续走新的形态。前面讲过,如果价格继续站在之前趋势最后一个波峰上,那么确认把握就大一些了。图中价格突破了上面的横线,而且下面成交量和持仓量持续放大,可以说各个方面都非常配合。\textbf{W 底部是否需要右低点比左低点高?}}
\subsection{趋势中继形态}