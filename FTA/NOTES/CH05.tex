\chapter{量价关系在期货实战中的运用}
期货中有成交量和持仓量,尤其是持仓量,更真实地反映了市场资金进出及市场人气的变化。但我们通常不会用其来作为一个过度预测未来行情的指标,其更大的应用价值是验证当下行情发展的持续性,一切都是为了让我们更好地理解趋势的运行特征。
\section{期货中的成交量和持仓量}
往往行情的启动是需要有大资金推动的,这时候持仓量的明显增加或者减少就显得格外重要了。

\begin{description}
    \item[国内交易所持仓信息披露]我们通常主要看前 20 名期货公司会员持仓的对比情况,从中可以看出期货市场阶段行情资金分布的特点。比如,多方持仓如果远大于空方,则说明多方主力目前明显强势,空单可能持仓分布相对分散,反之亦然。
    \item[美国 CFTC 持仓报告分析]非商业性持仓:一般认为非商业头寸是基金投机持仓,不涉及现货业务。在当今国际商品期货市场上,由于基金持仓变化比较频繁,是推动行情的主力,所以非商业性持仓是重点关注对象。商业性持仓:一般认为商业头寸与现货商有关,是套期保值者。还有一类套利持仓,同时持有同一个品种多头头寸和空头头寸的交易者的净持仓,视为套利单,计入此项。需要注意的是,多头持仓和空头持仓不包含套利持仓。

    综合上述信息,我们来看一下多空持仓的计算方式。在非商业投机头寸中,多头持仓和空头持仓都是指净持仓数量。比如,某交易商同时持有 4000 手多单和 2000 手空单,则其 2000 手的净多头头寸将归入“多头”,2000 手双向持仓归入“套利”持仓。所以,$\text{多头总持仓}=\text{非商业多头持仓}+\text{套利}+\text{商业多头}$;$\text{空头总持仓}=\text{非商业空头持仓}+\text{套利}+\text{商业空头}$。
\end{description}

关注投机性净持仓是比较重要的。真正套保性质的持仓一般不会轻易改变头寸方向,甚至进入交割月进行交割。净持仓的增减能够对行情产生直接影响,但通常更多的是对中短期行情发展的一个验证,并不能把持仓分析当成长期趋势判断的指标。