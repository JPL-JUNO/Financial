\chapter{量价关系在期货实战中的运用}
期货中有成交量和持仓量,尤其是持仓量,更真实地反映了市场资金进出及市场人气的变化。但我们通常不会用其来作为一个过度预测未来行情的指标,其更大的应用价值是验证当下行情发展的持续性,一切都是为了让我们更好地理解趋势的运行特征。
\section{期货中的成交量和持仓量}
往往行情的启动是需要有大资金推动的,这时候持仓量的明显增加或者减少就显得格外重要了。

\begin{description}
    \item[国内交易所持仓信息披露]我们通常主要看前 20 名期货公司会员持仓的对比情况,从中可以看出期货市场阶段行情资金分布的特点。比如,多方持仓如果远大于空方,则说明多方主力目前明显强势,空单可能持仓分布相对分散,反之亦然。
    \item[美国 CFTC 持仓报告分析]非商业性持仓:一般认为非商业头寸是基金投机持仓,不涉及现货业务。在当今国际商品期货市场上,由于基金持仓变化比较频繁,是推动行情的主力,所以非商业性持仓是重点关注对象。商业性持仓:一般认为商业头寸与现货商有关,是套期保值者。还有一类套利持仓,同时持有同一个品种多头头寸和空头头寸的交易者的净持仓,视为套利单,计入此项。需要注意的是,多头持仓和空头持仓不包含套利持仓。

    综合上述信息,我们来看一下多空持仓的计算方式。在非商业投机头寸中,多头持仓和空头持仓都是指净持仓数量。比如,某交易商同时持有 4000 手多单和 2000 手空单,则其 2000 手的净多头头寸将归入“多头”,2000 手双向持仓归入“套利”持仓。所以,$\text{多头总持仓}=\text{非商业多头持仓}+\text{套利}+\text{商业多头}$;$\text{空头总持仓}=\text{非商业空头持仓}+\text{套利}+\text{商业空头}$。
\end{description}

关注投机性净持仓是比较重要的。真正套保性质的持仓一般不会轻易改变头寸方向,甚至进入交割月进行交割。净持仓的增减能够对行情产生直接影响,但通常更多的是对中短期行情发展的一个验证,并不能把持仓分析当成长期趋势判断的指标。

\section{量价配合的常见情况}
我们对成交量、持仓量和价格的关系做一个基本梳理,具体内容如下。
\begin{enumerate}
    \item 成交量增加,持仓量也增加,价格上涨,表明多头主动开仓,从而推动价格上涨,可做多。
    \item 成交量萎缩,持仓量下降,若此时价格下跌,则表明多头平仓离场,多头持续动能不足,可做空。
    \item 成交量增加,持仓量也增加,价格下跌,则表明空头主动开仓,可做空。
    \item 成交量下降,持仓量也下降,价格上升,则表明空头平仓,空头持续动能不足,可做多。
\end{enumerate}

\begin{table}
    \centering
    \caption{量价配合的常见情况}
    \begin{tabular}{rrrr}
        \hline
        成交量 & 持仓量 & 价格 & 多空操作 \\
        \hline
        上升  & 上升  & 上升 & 多开   \\
        下降  & 下降  & 下跌 & 多平   \\
        上升  & 上升  & 下跌 & 空开   \\
        下降  & 下降  & 上升 & 空平   \\
        \hline
    \end{tabular}
\end{table}

另外,还有两种情况:
\begin{enumerate}
    \item 成交量增加,持仓量减少,价格上涨。该组合表示空方主动平仓。若出现在底部,其特征是价格小幅上涨,因为价格下跌到底部,空方心态较好,大部分已经盈利离场,落袋为安,而多方心态恐惧,不会马上大量接盘。但如果出现在顶部,空头为“逃命”平仓出货信号,而多头只是在高位挂单被动平仓,不存在主动打压力量,从而表现出价格大幅上涨的特征。
    \item 成交量增加,持仓量减少,价格下跌。该组合显示多空双方短期交易兴趣较浓,但空方不愿意继续增仓,而持仓量减少和价格下跌的主要动力来自于多头止损,显示多头急于平仓,追着价格卖出平仓。因此,通常在价格大幅下跌后。一旦持仓量开始增加,则显示新多资金开始介入和之前认为平错了的多头返身重新杀入,价格有可能会大幅反弹,容易形成 V 形反转形态。
\end{enumerate}

持仓量不变,价格下跌,表明是多换,高价的多头平仓,在更低的价格开多头。

\section{量价变化中的实战应用}
在反转行情开始的做多行为都是试仓,行情从一个市场结构完全切换到另外一个结构,除 V 形反转外,都需要一个比较长的过程。大部分情况下是进入震荡走势,如果持仓没有明显出现减少,那么行情继续持续的可能性就比较大。持仓明显减少的标准一般为下降 20\% 以上。

总体来看,在期货中持仓量更重要一些。持仓量主要应用于对一波行情走势的验证判断,指标工具更多地应用于跟踪行情的方向,持仓分析从另外一个角度揭示了行情运行的资金结构、行情节奏的变化及趋势发展的可靠性。

\figures{fig5-2}{动力煤期货主力日线合约。这是一个非常典型的行情走势,我们发现在价格持续上行的过程中经历了持仓最开始的增加和后面的持续回落,但价格始终持续走高。如果我们简单用持仓增减来预测价格下一步的走势则比较困难,但重点在于可以分析出市场的资金行为。前面的行情之所以启动,主要是因为有资金推动发起,多头资金占了绝对优势。当行情走了一个阶段后开始出现回落,前期运行节奏被打乱,主要是由于前面多头资金盈利落袋出场,当然也会有空头的止损发生,资金持仓总体出现减少。后面价格震荡后又出现了新的突破,持仓同时有一波短暂的上涨,随后缓慢持续回落。综合分析来看,在行情上涨的过程中,空头在持续被动止损,多头主力意图显然,过程中间还伴随着一些诱空的 K 线,在这段行情中,反复做空的资金会相当被动。目前该合约持仓已经逐渐减少到这波行情启动时的量,我们预计后面持续上行行情可能会暂时告一段落。}