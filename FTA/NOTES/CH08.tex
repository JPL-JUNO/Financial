\chapter{期货实战之短线交易}
\section{日内短线交易}
日内交易同样要符合期货亏小赚大的基本逻辑原理,在这个基础上,我们总结了更为简单的日内概率交易模式。以开盘 5 分钟 K 线为依据,5 分钟收阳线当日选择方向开多,5 分钟收阴线当日选择方向开空。不同品种设置不同的固定止损,可以按照品种 10 日内的平均波动幅度来设置。比如,螺纹 10 日内波动在 110 点左右,那么我们可以设置 30-40 点的止损,这样实际上当日只有一次交易机会,或者触发止损,或者我们设置到收盘前平仓。严格执行日内交易,不过夜,比如到下午 2 点 50 或者 2 点 55 平仓。我们称这样的方式为傻瓜交易模式。这种模式虽然比较简单,甚至可以说是粗糙,但给我们提供了一种很好的交易模式和思路。

日内开盘 5 分钟的涨跌好像和全天涨跌并不具备太强的逻辑上的关系,但从概率意义上讲,具备一定的操作价值,也完全符合限制亏损、放开利润空间的原则。
\begin{tcolorbox}
    之前有看过一种说法:开盘价是由交易日之间的散户决定的,而收盘价则是由多空经过一天的博弈决定的。我也认同这种讲法,并且认为收盘价是更加重要的价格。因此这里改进一下,如果收盘最后五分钟是阳线,那么直接成交做多,下一交易日开盘第五分钟平仓。
\end{tcolorbox}
\section{期货日常训练}
交易系统在某一个阶段出现不利,连续亏损一般是难以避免的,每个交易者或多或少都有过这样的经历,但由于每个人对止损的认识程度不一样,很多时候很容易造成激进加仓从而导致大亏,想停下来并不容易。这要从思想认识到交易行为有一个全面的训练和适应过程,并且我们止损是为了控制风险,而不是为了止损而止损。止损也不是盲目的,你的止损位的设置要和你的交易风格匹配,要可以完全说服自己认识到自己当下的交易出现了错误,所以才需要止损。从这个角度看,交易本质上除逻辑性外,还有概率性。不谈逻辑入不了交易系统的门,不谈概率也无法认识到交易的本质。

还有一个因素需要注意,即要排除过度学习的干扰。笔者接触过一些交易低迷时期的交易者,他们之所以能够翻身让账户创新高,第一因素就是先停下来,反省总结一下。然后马上就会发现不是自己的交易系统出了问题,而是自己本身的执行或心态出现了问题。很多时候只需要稍做调整,或者更多时候应该等一等,而不是频繁交易,更不需要再大量接触其他信息。