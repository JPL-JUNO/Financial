\chapter{期货实战成败的关键:资金盈亏的管理}
\section{如何设置止损}
最好的止损就是在技术关键支撑、阻力位置的止损。如果市场结构本身具有合理性,那么用这样的结构的关键位置来止损,甚至是资金加减仓的管理就很合理了。
\subsection{结合资金波动幅度设置止损}
有时候技术关键位置不太容易找到或者离进场位置很远,比如在强烈上行趋势中我们经常追高操作,当然我们可以在不同级别上去确定新的止损,单笔亏损可以承受的资金幅度也是一个应该考虑的因素,毕竟我们可能会出现连续亏损的情况,这时候单次亏损过大的话,账户是无法承受总体回撤波动的。

这里有一个简单实用的 2\% 原则,可以帮助我们定损和定量,就是在没有盈利的情况下初始本金单笔最大亏损不超过本金的 2\%。

这里有一个止损思想,就是合理设置止损是为了尽量不轻易被触发,而不是为了很容易被触发,应该属于预防性思维。如果我们本次交易出错了,我们有一个底线保护,同时止损的触发也说明行情比我们最初判断的的确出现了问题,证明我们判断错误。中长线交易中不是为了止损而止损,而是为了更好地追踪行情。相对短线来说,行情不利的时候可能止损会触发得更频繁一些,但由于可以设置更小的幅度,对我们本金的伤害会更小一些。

如果最近 5 天的 K 线长度相当,则可以用 5 日平均波动幅度来作为止损参考;如果有一根 K 线特别长或者特别短,我们就用 10 日 K 线平均波动来考察,从而尽可能减小异常波动带来的影响。这里的波动幅度是指每根 K 线的最高价和最低价之间的差异。
\section{仓位管理与止盈}
经统计,期货交易中经常会出现 3-5 天现象,即行情启动后无论是上涨还是下跌,一般最快速的那一个运行阶段可能就是一周左右,然后价格会面临大幅度的反向回调。毕竟很顺畅的大行情并不是经常会有的,但这样周阶段的波动行情一般是每个月都会出现的,属于常态行情。
\figures{fig7-6}{我们重点观察布林线与价格的位置关系,一般大涨行情价格会持续站在布林线上轨之上运行,大跌行情价格会持续站在布林线下轨之下运行,这样的行情很难得。波段行情中,通常布林线上、下轨会形成明显的支撑和压力区域,如果价格过度偏离,则会出现回归。图价格突破后第二天 K 线出现回落,我们可以部分盈利出场,后面行情发展进入了震荡阶段。实际上,这段行情中我们比较确定操作的就是 1-2 天的突破走势。}

\figures{fig7-9}{按照我们的方法,通常第一次开仓不以突破为依据,在价格线下面形成一个明显低点后可以初步\textbf{试多}进场。\textit{MACD 连续可以支持我们开仓做多,将 73350 作为做多止损价。}价格很快达到了布林线上轨,一旦遇到这样的行情,我们是有很大优势的。进场的优势被完全确认了,接下来就是赚多赚少的问题了。注意,这里突破后在布林线上轨出现了比较大幅度的震荡,这取决于我们每个人的选择,可以采取突破加仓方式,忍受一定程度的回落,或者观望。因为价格运行并没有跌破前面高点的平台,没有形成所谓的假突破形态。后面价格顺着布林线上轨再次突破,基本确认行情的持续性,我们可以再次加仓。}

\figures{fig7-12}{图中突破的位置我们设定为整数位 9000 点,此时多单进场,下面最明显的关键技术位置是 8685 点,我们以这个为最佳止损位置。行情启动后上行很快,最后的行情暴涨达到了 9845 的高点,然后快速回落 415 点的幅度,最低到 9430 点的位置。我们可以按照前面讲的布林线上轨方式提前止盈,但如果跟踪利润的发展,我们也可以只看 K 线结构。我们设置的初始止损幅度是 315 点。价格上涨到 9315 点达到了基本盈亏平衡点,如果价格继续发展,我们可以按照这个倍数来考察,如再上涨一个盈亏平衡点是 9630 点,依次上推为 9945 点。价格最高发展到 9845 点,在三倍盈亏平衡点之内,我们可以接受价格回调到 2 个盈亏平衡幅度内,也就是 9630 点附近。很显然,价格大幅度回落,我们提前止盈了。后面根据价格发展完全可以再择机进场。}

总之,我们的原则是总盈利大于总亏损,在总体盈亏比大于 1 的情况下,我们根据行情的发展可以有更多止盈的选择。通常来看,在强势上行行情中出现大幅度回落,说明上行人气暂时已散,价格一般需要时间休整才可以发动新的行情,多空主力博弈都暂时告一段落。