\chapter{其他反转现象}
\section{扩散形态}
\textbf{扩散形态}被称为“倒转三角”,\textit{因为它们一开始时出现窄幅震荡,然后振幅不断扩大,形成两条发散的界线。}

在对扩散形态出现后的股价走势追踪研究了 20 多年后,我们认为,扩散形态无疑是熊市信号,当然也不完全排除后续股价进一步上涨的可能,但这种上涨只是最后的疯狂。如果你看到这个形态,不但不应该追高,而且应迅速卖出或换股。
\section{扩散形态的成交量}
三角形态和扩散形态的另一个区别在于成交量的变化。在真正的三角形态中,成交量不断缩小,通常第一次短线反转都伴随着高成交量,之后随着股价向三角形的顶点运动,成交量逐渐萎缩。股价突破三角形后,成交量开始扩大。如果是向上突破,成交量则会立刻明显放大。而在扩散形态中,成交量往往全程维持在高位,且其变化无规律可循。如果扩散形态出现在涨势后(大多数情况),那么不仅第 1 次短线反转的成交量很高,第 2 次反弹、第 3 次反弹,甚至在多个短线底部的成交量都会很高。所以,股价和成交量的走势都呈现出无序的剧烈波动。
\section{标准扩散顶}
标准扩散顶有 3 个连续升高的高点,在它们之间有两个低点,第 2 个低点比第 1 个低。一旦股价从第 3 个高点回调至第 2 个低点以下,该形态就宣告完成,预示着股价走势反转。

\figures{10-2}{该股在走出一个扩散顶形态后,于当年 10 月结束了这轮牛市,图上用数字 1-5 标出了该扩散顶的重要转折点。只有在股价从第 2 个短线低点(点 4)开始上行后,扩散形态才能被识别;此时,点 3 高于点 1,点 4 低于点 2。股价在到达点 5(a 和 b)后,调头下行并在 B 点有效突破(比点 4 低近 6 \%),扩散顶宣告完成,预示着长线趋势反转。在这个案例中,扩散顶对反转的预示效力极强,因为如我们的图表所示,短短 4 周内(10 月 18 日至 11 月 14 日),空气精炼公司股价从 220 多美元跌破了 80 美元,后来直到 1932 年才见底。上面这个经典案例中,有一些细节需要注意。首先,第 3 个高点形成于 5a,随后的回调止步于 195 美元(明显高于点 4),接着股价重拾升势。但是,这个案例提醒我们对扩散顶要谨慎;只有当股价明显超过前一个高点后,才能信任扩散顶。图中空气精炼公司股价在 5b 短暂触及了 223 美元,比 5a 高了 2 美元但不到 3\%,且当天收盘价在 5a 以下。10 月 24 日股价突破到达 B,比点 4 低了 3\% 以上。此时出现了典型的扩散顶形态:股价反弹至 P ,收复最近一个高点(5b)和首次突破后的低点(B)之间失地的一半左右。根据我们的经验,扩散顶中出现这样一次反弹的概率至少为 80\%,反弹幅度一般为先前跌幅的一半,但也可能高达 2/3。}

\section{楔形}
楔形的图形特征是:\textit{股价波动由两条收拢的直线(或近似于直线的曲线)界定。}\textbf{但与三角形态不同的是,这两条直线同时向上或向下倾斜。}在对称三角形中,上界线向下倾斜,下界线向上倾斜;在直角三角形中,一条界线向上或向下倾斜,另一条保持水平。而在上升楔形中,两条界线都向上倾斜,但因为这两条线最终会相交,所以下界线倾斜的角度大于上界线;在下降楔形中则相反。

你或许会认为,既然上升三角形(即一条界线水平一条界线向上倾斜的形态)预示着牛市,那么上升楔形(即两条界线同时向上倾斜)更应该预示着大牛市的到来。然而事实并非如此。前文曾经提到过,上升三角形的平顶代表股票正在以特定价位出手,当这些卖盘被全部吸收后(上倾的下界线代表卖盘迟早会被吸收),股价就会上涨。\textbf{而在上升楔形中,没有明显的出货价位,但买盘会逐渐耗尽,每一波涨势都比上一波更无力,最终趋势反转。所以,上升楔形预示着一个越走越弱的趋势。}

上升楔形与正常上升通道的区别在于,楔形对上攻构成了某种限制。楔形的两条界线收拢于一点,股价将止步于该点附近,然后回调。

楔形一般需要 3 周以上时间来完成;若短于 3 周,则最好归类为“三角旗形”。从楔形的起点线(两条界线开始收拢时)到楔形的顶点(两条界线的相交点),股价在界线内波动往往会达到至少 2/3 的长度;许多时候,股价上升到楔形顶点的水平,甚至在该水平之上最后反弹一波后才开始暴跌。而股价一旦跌破楔形的下界线,通常就会义无反顾地急跌,回吐先前楔形的全部涨幅,有时甚至跌去更多。楔形的成交量变化常与三角形一样,随着股价向楔形的顶点攀升,成交量逐渐缩小。
\section{下降楔形}
上升楔形倒转过来,就成了下降楔形。但是,下降楔形完成后的股价走势有自己的特点。股价突破上升楔形后,往往迅速下跌;而股价突破下降楔形后,更倾向于先呈现一段横盘走势或交投清淡的碟形走势,之后再上涨。因此,\textit{面对上升楔形,投资者应果断抛售,落袋为安;而面对下降楔形,投资者不妨静观其变,持股待涨。}

如果一条界线接近于水平,或者每日收盘价基本处于同一水平线上,那么不妨将形态视为三角形。
\section{常见于熊市反弹的上升楔形}
最后,我们补充一点:\textbf{上升楔形是熊市反弹中的常见形态。}市场在经历了长期下跌后常会出现楔形,以至于大家会问牛市是否已经到来。在长线熊市以头肩底结束时,其左肩至颈线的这波涨势常常会以上升楔形呈现。等差周线图中的上升楔形几乎总出现于熊市,反映上涨动能在减弱,这也是熊市反弹的基本特征。
\section{单日反转}
单日反转常见于恐慌性抛盘的末期,被称为\textit{抛售高峰日}。但是,首先,什么是单日反转呢?

\begin{tcolorbox}
    首先,这一天的成交量极高,远超过去几个月内任何一天的成交量。在此之前,股价已经经历了长时间的稳步上涨(或持续下跌),成交量也不断放大。当天开盘后,股价立即飙升或急跌,势不可挡。而且,这一天的开盘价常常会远高于(或低于)前一天的收盘价,在图表上留下一个明显的缺口。在涨势(或跌势)停止之前,股价已经在一两个小时内完成了一般需要三四天才能达到的涨/跌幅。但这波涨势(或跌势)迟早会止步,可能是开盘1小时后,也可能是快收盘时。之后,股价小幅波动,交投持续活跃。然后,趋势突然反转,股价迅速朝反方向运动。尾盘时,交投大增,将股价压低(或拉升)至当日的开盘价附近。总体看来,这一天成交量骤增,股价盘中波动达 2\%-3\%,但收盘价与前一个交易日相差不大。
\end{tcolorbox}

顶部单日反转常见于流通盘较小的股票。这些股票经过一段时间的稳定上涨,吸引了大量投资者。但这种反转形态基本不会在平均指数的图表上出现。抛售高峰日(底部单日反转)明显可见于平均指数出现恐慌性或异常下跌时。

单日反转不会影响长线趋势。但一般来说,新趋势(即收盘时的趋势)的效力并不强,股价通常会在收盘价附近徘徊一段时间,呈现出某种形态,之后再走出一段中线趋势。单日反转常常是意义更大的技术形态中的一部分,是我们预测趋势发展的重要线索。无论如何,一旦单日反转出现,我们就要更加小心,密切注意后续走势,准备迎接意义更重大的趋势。