\chapter{其他反转现象}
\section{扩散形态}
\textbf{扩散形态}被称为“倒转三角”,因为它们一开始时出现窄幅震荡,然后振幅不断扩大,形成两条发散的界线。

在对扩散形态出现后的股价走势追踪研究了 20 多年后,我们认为,扩散形态无疑是熊市信号,当然也不完全排除后续股价进一步上涨的可能,但这种上涨只是最后的疯狂。如果你看到这个形态,不但不应该追高,而且应迅速卖出或换股。
\section{扩散形态的成交量}
三角形态和扩散形态的另一个区别在于成交量的变化。在真正的三角形态中,成交量不断缩小,通常第一次短线反转都伴随着高成交量,之后随着股价向三角形的顶点运动,成交量逐渐萎缩。股价突破三角形后,成交量开始扩大。如果是向上突破,成交量则会立刻明显放大。而在扩散形态中,成交量往往全程维持在高位,且其变化无规律可循。如果扩散形态出现在涨势后(大多数情况),那么不仅第 1 次短线反转的成交量很高,第 2 次反弹、第 3 次反弹,甚至在多个短线底部的成交量都会很高。所以,股价和成交量的走势都呈现出无序的剧烈波动。