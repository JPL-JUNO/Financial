\chapter{迈克尔·马库斯:绝不重蹈覆辙}
\begin{tcolorbox}[title=在市场价格封死跌停前,还处下跌途中的时候,你考虑过清仓离场吗?]
    当时我认为我应当清仓离场,但我只是看着价格下跌。那时我人已完全瘫掉了,希望市场能有转机,价格能够反身向上。我就是目不转睛地看啊,看啊,当价格封死跌停后,我已无法离场。当天我苦思冥想了一整夜,但确实已无计可施,我没有更多的钱可以追加保证金,不得不清仓离场。第二天早上一开盘,我在开盘价处清空了全部头寸。
\end{tcolorbox}

交易时不要孤注一掷

追随趋势必须具有耐心。

\begin{tcolorbox}[title=那时你仍然在犯什么错呢,你是否还记得?]
    我认为,那时我缺乏足够的耐心,操之过急,没等事态明确、清晰就动手行动。
\end{tcolorbox}

\begin{tcolorbox}[title=在那时要完全做多?]
    是的。每一样商品的价格都在上涨中。虽然我交易得很好,但还是犯了一个极为严重的错误。在大豆期货的超级牛市中,当其期货价格从 3.25 美元涨到近 12 美元时,我一时冲动,清仓兑现了利润,于是失去宝贵的多头头寸(相当于失去了一切)。我没有追随趋势坚定持仓,而是用自己的想象和主观臆测取而代之。所以当我清仓离场时,艾迪·塞柯塔仍旧持有多头头寸待在场内。接着大豆期货的价格连续 12 个交易日达到涨停,看得我痛不欲生。我是一个争强好胜的人,每天我来到办公室,得知塞柯塔依然持仓不动,待在场内,不由想起我的清仓离场。我害怕去那上班,因为我知道大豆期货的价格又会涨停,而我再无重新建仓的机会。
\end{tcolorbox}

\begin{tcolorbox}[title=解决这一问题的方法是什么呢?这里的问题是以前关键点位被突破后就是趋势,但是现在很多假突破。]
    我认为秘诀在于减少交易的次数。最好的交易应该是以下三方面的情况都对你有利,也就是基本面、技术面以及市场的情况、基调。首先在基本面上,供给和需求应处于不平衡状态,这样才会导致市场价格的显著波动。其次,技术图表上所示市场价格的运动方向要与基本面所指示的价格运动方向一致。最后,当有消息、新闻传出时,市场价格所做出的反应要与该市况下的市场心理一致。例如,在牛市中会忽略利空的消息,而对利多的消息反应会非常强烈。如果你能精挑细选,只做全部符合这三方面要求的交易,那你在任何环境下、任何市场中赚钱都是肯定的。
\end{tcolorbox}

市场上的大玩家,包括政府在内,总会有摊牌的一天,总会有真相大白的一天。如果我们看到价格向不利于我们的方向突然运动,而且我们对此无法理解,搞不清楚,通常就清仓离场,等到以后再探寻其中的原因。

\begin{tcolorbox}[title=最为重要的是,你出场做得非常好。是什么东西向你发出警示,让你感觉价格已接近顶部(或底部)?]
    在当时,许多市场都非常疯狂。我的一条交易法则就是:当市场波动性变得很大,走势变得极为疯狂时,那就是我清仓离场的时候。我衡量时采用的一种方法就是涨跌停板的天数。在那时候,我们会碰到许多价格连续数日涨停的情况(当时是商品期货市场的大牛市)。当连续第三个涨停时,我会开始变得非常非常谨慎。我几乎总是在第四个涨停板清仓离场,并且如果在第四个涨停板,我还留下少许多头头寸,没有全部清仓的话,我在第五个涨停板时一定会全部了结,这是我的强制规定,一定要做到。面对波动性如此大的市场,我会强制自己离场。
\end{tcolorbox}

\begin{tcolorbox}[title=当你下达进场建仓的交易订单时,一定伴有清仓离场的交易订单?]
    是这样的。另外,如果你刚建仓的头寸看上去并不妙,此时不要认为改变看法、观点是一件令人尴尬的事,要马上改变看法,清仓离场。
    \tcblower
    如果你对持有的头寸开始不再确信,有所怀疑,不知该如何操作,那就清仓离场。你总会有再次进场建仓的机会。当你疑虑重重时,那就清仓离场,睡个好觉。你在这笔头寸上已耗时太多,待到明日,所有一切都会一目了然,即疑虑时先离场观望,待局势明了再进行操作。
\end{tcolorbox}

\begin{tcolorbox}[title=如果你已发现各市场或品种共有的价格行为,那么“一旦某上涨滞后的市场或品种率先开始下跌,你就做空该市场或品种”,这是否是必然的交易法则?]
    只要某个市场或品种的价格运动和其他所有相关市场或品种的价格运动出现背离,你完全能在该市场或品种上下注交易。当利多消息出台,某市场或品种的价格却无法上涨,那你一定要做空。
    \tcblower
    \textbf{玻璃、纯碱、螺纹钢存在很高的相关性,似乎可以反应这种现象。}
\end{tcolorbox}

有一种愚蠢的想法,那就是“阴谋论”,即认为市场可以人为操控,某些价格波动的背后隐藏着阴谋。我认识许多来自世界各地的杰出交易者,因此我能断言:在 99\% 的时间里,市场自身的力量要大过任何个人或机构的外力,市场迟早会向它该去的地方进发,市场自身形成的运动方向没人能够改变。有时或有例外,但这种例外不会持续太久。

\textbf{也许在 A 股的小市值中确实存在人为的控制。}

我认为,“连续亏损”就是亏损生亏损的过程,就是前面的亏损导致后面的亏损。当你开始遭遇亏损时,就会触发你的负面情绪和消极心理,并终将导致整个人的悲观和消极,从而影响接下来的交易。

某公司没有市场占有率极高的领域,即该公司的产品市场占有率都不高,那么对我来说,每股收益的高低已不再重要,即使它每股收益很高,我也不会选择它。但是如果公司产品的市场占有率确实不高,但每股收益却在不断增长,而且其产品所处的市场尚未达到饱和,发展空间较大,还有许多市场份额等待抢占,这样的股票对我来说,是具有吸引力的。

交易就是控制自己的情绪,察觉别人的情绪。交易要洞察大众贪婪和恐惧的心。无论在哪个市场,这些都是完全相同的。

如果你不能在可以大赚的交易上持盈,那么你就无法弥补止损累积起来的亏损。