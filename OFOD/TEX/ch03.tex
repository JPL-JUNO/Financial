\chapter{利用期货的对冲策略}
\section{基本原理}
\subsection{空头对冲}
\textbf{空头对冲}(short hedge)是指对冲者选择期货的空头方。当对冲者已经拥有了某种资产并期望在将来某时刻卖出资产时,这时选择期货空头对冲是合理的。在当前不拥有资产,但在将来会拥有资产的情况下,也可以选择空头对冲。
\subsection{多头对冲}
持有期货多头的对冲策略叫\textbf{多头对冲}(long hedge)。当公司已知在将来需要买入一定资产并想在今天将价格锁定时,可以采用多头对冲。
\section{拥护与反对对冲的观点}
\subsection{对冲与竞争者}
在某些行业里,如果对冲并不是常规做法,这时一家公司选择与别人都不相同的做法也许没有太大意义。行业之间的竞争压力会迫使企业调整商品和服务价格来反映原材料价格、利率、汇率等的变化。因此,一家选择不对冲风险的企业可能期望利润率基本保持不变,而一家选择对冲风险的企业可能期望利润率会有很大的浮动。比如说你是一家黄金加工商,对冲锁住黄金的成本,但是黄金产品和黄金价格时高度相关的,那么如果黄金价格下跌,黄金产品价格下跌,这种情况下对冲将导致成本变高,收益减少,对冲副作用。

为了说明这一点,我们考虑两家黄金珠宝加工商—— SafeandSure(保守)公司与 TakeaChance(冒险)公司。我们假定在这个行业中大多数公司不采用对冲措施,TakeaChance 也不例外。但是 SafeandSure 却采用了与众不同的政策:公司决定利用期货合约来锁定未来18个月内的黄金价格。当黄金价格上涨时,经济压力会造成珠宝批发价格普遍上涨,这样一来,TakeaChance 的利润会基本不受影响。与此相反,由于采用了对冲,SafeandSure 公司的利润将会提高。而当黄金价格下跌时,经济压力会造成珠宝批发价格普遍下跌,从而 TakeaChance 公司的利润仍然不会受到太大影响,但 SafeandSure 公司利润却将会下降。在某些极端条件下,由于对冲的缘故 SafeandSure 利润可能会出现负值。在 \autoref{tbl3-1} 总结了此例。
\begin{table}
    \centering
    \caption{在同业竞争对手不对冲时采用对冲策略的危险性}
    \label{tbl3-1}
    \begin{tabular}{cccc}
        \hline
        黄金价格的变化 & 对黄金珠宝价格的影响 & 对 TakeaChance 利润的影响 & 对 SafeandSure 利润的影响 \\
        \hline
        上涨      & 上涨         & 没有影响                & 增加                  \\
        下跌      & 下跌         & 没有影响                & 减少                  \\
        \hline
    \end{tabular}
\end{table}
\subsection{对冲可能会使结果更糟}
我们应该认识到这样一个现实问题:与不采用对冲相比,采用对冲既可以增加也可以减少企业的利润。
\section{基差风险}
在实际中对冲常常没这么容易,部分原因如下:
\begin{enumerate}
    \item 需要对冲价格风险的资产与期货合约的标的资产可能并不完全一样。
    \item 对冲者可能无法确定买入或卖出资产的准确时间。
    \item 对冲者可能需要在期货到期月之前将期货平仓。
\end{enumerate}
这些问题就引起了所谓的\textbf{基差风险}(basis risk)。
\subsection{基差}
在对冲意义下,基差(basis)的定义如下\footnote{这是通常的定义,有时也采用另外一种定义(尤其当期货是关于金融资产时):$\text{基差}=\text{期货价格}-\text{现货价格}$。}
$$\text{基差}=\text{被对冲资产的现货价格}-\text{用于对冲的期货合约价格}$$
如果被对冲的资产与期货合约的标的资产相同,在期货到期时基差应当为 0。在到期日之前,基差可正可负。

随着时间的变化,现货价格变化与特定月份期货的价格变化并不一定相同,因而会导致基差的变化。当基差变大时称为\textbf{基差增强}(strengthening of the basis),当基差变小时称为\textbf{基差减弱}(weakening of the basis)。\autoref{fig3-1} 中的基差为正,图形展示了在期货到期之前基差随时间变化的形式。
\figures{fig3-1}{基差随时间的变化}

注意,基差风险可以使得对冲者的头寸得到改善或导致恶化($F_1+b_2$)。假设一家公司计划在将来卖出资产,并决定采用空头对冲,如果基差在意想不到的情况下增强(即增大),那么对冲者的头寸会有所改善,这是因为在考虑了期货的盈亏之后,卖出资产时会拿到更好的价格;如果基差在意想不到的情况下减弱(减小),那么对冲者的头寸会有所恶化。同样,假设一家公司计划在将来要买入资产,决定采用多头对冲,如果基差在意想不到的情况下增强,对冲者的头寸将会恶化,这是因为在考虑了期货的盈亏以后,买入资产时要支付更高的价格;如果基差在意想不到的情况下减弱,对冲者的头寸将会得到改善。
\begin{tcolorbox}
    如果基差在意想不到的情况下增强(即增大)。这样的表述应该不对,如果平仓是基差不是 0,那么正的基差有利于期货的 short,负的基差有利于期货的 long。
\end{tcolorbox}
有时给对冲者带来风险的资产与用于对冲的合约标的资产并不一样,这种情形下的对冲叫\textbf{交叉对冲}(cross hedging)。在这种情况下,基差风险一般会更大。定义 $S_2^*$ 为期货合约标的资产在时刻 $t_2$ 的价格。与上面相同,$S_2$ 是被对冲资产在时刻 $t_2$ 的价格。通过对冲,公司确保购买或出售资产的价格为
$$F_1+S_2-F_2$$
上式可变形为
$$F_1+(S_2^*-F_2)+(S_2-S_2^*)$$
$S_2^*-F_2$ 和 $S_2-S_2^*$ 代表基差的两个组成部分:$S_2^*-F_2$ 代表当被对冲资产与期货合约标的资产一致时,对冲所产生的基差;$S_2-S_2^*$ 是由于被对冲资产与期货合约标的资产不一样而产生的基差。
\subsection{对合约的选择}
影响基差风险的一个关键因素是选择用来对冲的期货合约。这里的选择包括两部分:
\begin{enumerate}
    \item 对期货合约标的资产的选择;
    \item 对交割月份的选择。
\end{enumerate}
如果被对冲的资产刚好与期货的标的资产相同,这里的第一个选择一般会很容易。在其他情形下,对冲者必须通过仔细分析来确定哪一种期权价格与被对冲资产的价格有\textbf{最紧密的相关性}。

\textbf{一般来讲,当对冲的期限与期货交割月份之间的差距增大时,基差风险也会随之增大}。一种经验法则是尽量选择与对冲期限最近,但在其之后的交割月份。假定某一资产上期货的到期月分别为 3 月、6 月、9 月和 12 月。对于在 12 月、1 月、2 月到期的对冲,应当选择 3 月的合约;对于在 3 月、4 月、5 月到期的对冲,应当选择 6 月的合约,等等。这种经验法则假设了所有满足对冲需要的合约都有足够的流动性。
\section{交叉对冲}
\textbf{对冲比率}(hedge ratio)是指持有期货合约的数量与资产风险敞口数量的比率。当期货标的资产与被对冲资产一样时,对冲比率当然应该取为 1.0。

当采用交叉对冲时,将对冲比率取为 1.0 并不一定是最优的选择。对冲者采用的对冲比率应当使被对冲后头寸价格变化的方差达到极小。
\subsection{计算最小方差对冲比率}
我们首先提出一个假设:期货市场没有每日结算制度。最小方差对冲比率取决于现货价格的变化与期货价格变化之间的关系。我们采用以下符号:

$\Delta S$:在对冲期限内,现货价格 $S$ 的变化;

$\Delta F$:在对冲期限内,期货价格 $F$ 的变化。

假设 $\Delta S$与 $\Delta F$之间为近似线性关系,可以表示为:
$$\Delta S=a+b\Delta F+\epsilon$$
假设对冲比率是 $h$(即,现货风险敞口 $S$ 的百分比 $h$ 通过期货对冲)。那么,现货价格每变动 1 单位,头寸价值的变化为:
$$\Delta S-h\Delta F=a+(b-h)\Delta F+\epsilon$$
可知当 $h=b$ 是方差最小。根据线性回归中的斜率式可知:
\begin{equation}\label{eq3-1}
    h^*=\frac{Cov(\Delta S, \Delta F)}{Var(\Delta F)}=\rho\frac{\sigma_{\Delta S}}{\sigma_{\Delta F}}
\end{equation}
\autoref{eq3-1} 显示最佳对冲比率等于 $\Delta S$ 与 $\Delta F $ 之间的相关系数乘以 $\Delta S$ 的
标准差与 $\Delta F $ 的标准差之间的比率。

\textbf{对冲效率}(hedge effectiveness)可以定义为对冲所消除的方差量占总方差的比例,这正是将 $\Delta S$ 对 $\Delta F$ 进行线性回归的 $R^2$ 系数,等于 $\rho^2$。

\subsection{最优合约数量}
为了计算对冲所用的合约数量,定义:

$Q_A$:被对冲头寸的数量(单位数量);

$Q_F$:一份期货合约的规模(单位数量);

$N^*$:用于对冲的最优期货合约数量。

期货合约应当是关于 $h^*Q_A$ 单位的资产,因此所需要的期货合约份数为
\begin{equation}\label{eq3-2}
    N^*=\frac{h^*Q_A}{Q_F}
\end{equation}
\subsection{每日结算的影响}
如果我们用来对冲的是远期合约,那么上面的分析是正确的。当利用期货进行对冲时,合约的每日结算意味着有一系列的一天对冲(而不是只有一个对冲)。定义:

$\hat{\sigma}_S$:现货价格每天百分比变化的标准差;

$\hat{\sigma}_F$:期货价格每天百分比变化的标准差;

$\hat{\rho}$:期货价格和现货价格每天百分比变化之间的相关系数。

现货价格和期货价格变化的标准差分别为 $\hat{\sigma}_SS$和 $\hat{\sigma}_FF$是两者的相关性。从 \autoref{eq3-1} 可知,期限为 1 天的最优对冲比率为
$h^*=\hat{\rho}\frac{\hat{\sigma}_SS}{\hat{\sigma}_SF}$
\begin{tcolorbox}
    可以理解为一天对冲就是对冲一次,\autoref{eq3-1} 中的 $\Delta S\approx \text{百分比}\times S$
\end{tcolorbox}

因此,从 \autoref{eq3-2} 中得出用于对冲的最优期货合约数量为
$$N^*=\hat{\rho}\frac{\hat{\sigma_S}SQ_A}{\hat{\sigma_F}FQ_F}$$

\autoref{eq3-1} 中的对冲比率是现货价格实际变化对于期货价格实际变化做线性回归时的最优拟合直线的斜率。另一种对冲比率 $\hat{h}$ 可以通过计算现货价格每天百分比变化对期货价格每天百分比变化做线性回归时的最优拟合直线的斜率得出:
$$\hat{h}=\hat{\rho}\frac{\hat{\sigma_S}}{\hat{\sigma_F}}$$
然后
\begin{equation}\label{eq3-3}
    N^*=\frac{\hat{h}V_A}{V_F}
\end{equation}
其中 $V_A=SQ_A$ 是被对冲的头寸的价值,$V_F=FQ_F$ 是期货价格与一份期货合约数量的乘积。

从理论上讲,当现货价格和期货价格变动时应当调整合约的数量,但在实际中最优头寸在每天中的变化很小,常常忽略不计。

通过通盘考虑对冲剩余期限内所得或付出的利息后,可以进一步改进上面的分析。假设在时间 $t$ 时可以计算从 $t$ 到对冲结束这段时间里所得或付出的利息为 5\%,且对冲的剩余期限为 1 年,在时间 $t$ 时应当将 $N^*$ 除以 1.05。\textbf{可以理解为 $V_A$ 是未来时刻,而 $V_F$ 是 $t$ 时刻。}

由于每日结算的影响而对 \autoref{eq3-2} 所做的改善称为\textbf{尾随对冲}(tailing the hedge)。
\section{股指期货}
\subsection{股票组合的对冲}
股指期货可用于对冲风险分散良好的股票投资组合。定义:

$V_A$:股票组合的当前价值;

$V_F$:一份期货的当前价值(定义为期货价格乘以期货规模)。

如果组合是为了跟踪股票指数,这时的最优对冲率 $h^* $为 1.0,由 \autoref{eq3-3} 得出需要持有的期货空头合约数量为
\begin{equation}\label{eq3-4}
    N^*=\frac{V_A}{V_F}
\end{equation}

当股票组合不跟踪股指时,我们可以采用资本资产定价模型(CAPM)。资本资产定价模型中的参数 $\beta$ 是将组合超过无风险利率的收益与指数超过无风险利率的收益进行回归所产生的最佳拟合直线的斜率。

一个 $\beta$ 值等于 2.0 的组合对市场的敏感度是一个 $\beta$ 值等于 1.0 的组合的两倍。因此,为了对冲这一组合,我们将需要两倍数量的合约。类似地,一个 $\beta$ 值等于 0.5 的组合对市场的敏感度是一个 $\beta$ 值等于 1.0 的组合的一半,因此我们只需要一半数量的合约来对冲风险。一般来讲

\begin{equation}\label{eq3-5}
    N^*=\beta\frac{V_A}{V_F}
\end{equation}
在这个公式中,我们假设期货合约的到期日与对冲期限很近。

将 \autoref{eq3-3} 与 \autoref{eq3-5} 比较,可以得出 $h^*=\beta$,该等式对我们来讲并不意外,对冲比率是将组合价值在一天内的百分比变化对于股指期货价格在一天内的百分比变化做线性回归时的最优拟合直线的斜率,$\beta$ 是组合变化对股指变化做线性回归时的最优拟合直线的斜率。
\subsection{对冲股权组合的理由}
人们自然要问,为什么要采用期货合约来对冲呢?要是只为了取得无风险利率的收益,对冲者只需要变卖资产,并将得到的资金投放于类似短期国债之类的无风险产品即可。

这个问题的一个答案是,如果对冲者认为组合中的股票选取得很好,那么实施对冲是有道理的。在这种情况下,对冲者对市场的整体风险不很确定,却确信组合中的股票收益会高于市场的收益(对组合 $\beta$ 值进行调整之后)。对冲者可以采用股指期货来消除因市场变动而触发的风险,从而使对冲者仅仅暴露于股票组合与市场的相对表现中。稍后我们将对此做进一步的讨论。另一个原因是对冲者计划在很长一段时间内持有股票组合,但需要在短时间内对市场的不确定性进行保护。将资产变卖并在将来买回的做法可能会触发太高的交易费用。
\subsection{改变组合的 $\beta$}
有时也可以利用期货合约将组合的 $\beta$ 调整到非零的值。

通常来讲,当将组合的 beta 从 $\beta$ 变为 $\beta^*$ 时,如果
$\beta > \beta^*$,所持期货空头的数量应当为
$$(\beta-\beta^*)\frac{V_A}{V_F}$$

如果 $\beta < \beta^*$,所持期货多头的数量应当为
$$(\beta^*-\beta)\frac{V_A}{V_F}$$
\subsection{锁定所选股票的优势}
假设你擅长挑选比市场表现更好的股票。你拥有一只股票或一个小的股票组合。你不知道在今后几个月内股票市场的表现将会如何,但你非常确认你持有的股票将比市场表现要好,你这时该如何做呢?

你应该持有数量等于 $\beta V_A/V_F$ 的股指期货合约空头,其中 $\beta$ 为你所持有股票的 beta 值,$V_A$ 为持有的股票价值,$V_F$ 为一份股指期货合约的价值。如果你的股票投资组合比一个具有同样 $\beta$,但风险分散很好的组合表现要好,你这时就会赚钱。

\begin{tcolorbox}
    这里说的好是指任何时候都要表现的好,下跌的时候跌的少,上涨的时候长得多。
\end{tcolorbox}
\section{向前滚动对冲}
有时对冲的期限要比所有能够利用的期货期限更长,这时对冲者必须对到期的期货进行平仓,同时再进入具有较晚期限的合约。这样可以将对冲向前滚动很多次。这种做法称为向前滚动对冲(stack and roll)。考虑某家公司,它希望利用期货空头来减少在将来时刻 $T$ 收到某个资产时所带来的风险。如果在市场上存在期货合约 $1,2,3,\cdots,n$(并不一定目前都存在),其到期日一个比一个更晚。公司可以采用以下策略:
在 $t_1$ 时刻:进入合约 1 的空头

在 $t_2$ 时刻:进入合约 1 平仓,进入合约 2 的空头

在 $t_3$ 时刻:进入合约 2 平仓,进入合约 3 的空头

...

在 $t_n$ 时刻:进入合约 $n-1$ 平仓,进入合约 $n$ 的空头

在 $T$ 时刻:对合约 $n$ 平仓

在实际中,公司通常在每个月中对标的资产都有风险敞口,并且利用一个月的期货合约来对冲,因为这样的短期限合约流动性较强。在最初时,公司会承约足够多的头寸(即将合约叠在一起)来覆盖所有在对冲持有期内的风险敞口。一个月过后,公司将所有期货合约平仓,并向前滚动,即承约新的一个月期合约来应对新的风险敞口,等等。

当制订对冲计划时,我们一定要将流动性会产生的潜在问题考虑在内。
\begin{tcolorbox}[title=业界事例:德国金属公司对冲的失误]
    有时采用向前滚动对冲的方式会对公司现金流造成压力。20 世纪 90 年代初,这一问题在一家德国公司,即德国金属(简称 MG)身上体现得淋漓尽致。

    MG 公司以高于市场价 6-8 美分的固定价格向其客户卖出了大量的 5-10 年民用燃料油和汽油合约,然后采用短期期货合约的多头来对冲其风险敞口,在对冲过程中,将这些短期合约向前滚动。随后原油价格下跌,因持有期货,MG 需要补充保证金,这给公司造成了短期现金流的压力。公司内设计这一对冲策略的人员认为这些短期现金流可以被长期固定价格合约的现金流抵消,然而公司高管与贷款银行对这笔巨额现金的流出十分担心,最后公司只好对所有对冲交易进行平仓,同时同意客户的要求而放弃了固定价格的合约。结果 MG 损失了 13.3 亿美元。
\end{tcolorbox}