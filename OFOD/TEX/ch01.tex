\chapter{导论\label{ch01}}
\section{远期合约}
一种比较简单的衍生产品是远期合约(forward contract),它是在将来某一指定时刻以约定价格买入或卖出某一产品的合约。远期合约可以与即期合约(spot contract)对照,即期合约是指立刻就要买入或卖出资产的合约,远期合约常常是金融机构之间或金融机构与其客户之间在场外市场进行的交易。

在远期合约中,同意在将来某一时刻以约定价格买入资产的一方被称为持有多头头寸(long position,简称多头),远期合约中的另外一方同意在将来某一时刻以同一约定价格卖出资产,这一方被称为持有空头头寸(short position,简称空头)。
\subsection{远期合约的收益}
一般来讲,在合约到期时,对于远期合约多头方来讲,每一单位合约的收益为
$$S_T-K$$

这里 $K$ 为合约的交割价格(delivery price),$S_T$ 为资产在合约到期时的市场价格,合约中的多头方必须以 $K$ 价格买入价值为 $S_T$ 的资产。同样,对于远期合约的空头方来讲,合约所带来的收益为
$$K-S_T$$
以上所列的两项收益均可正可负。因为签订远期合约的费用为 0,所以合约的收益也就是交易员所有的盈亏。

假设企业有义务在 6 个月后以 1223000 美元价格买入 100 万英镑。当汇率上涨时,假如在 6 个月后1英镑值 1.3000 美元,这时对企业来讲,远期合约价值为 $+77000(=1300000-1223000)$ 美元。远期合约保证企业可以按每英镑 1.2230 美元(而不是 1.3000 美元的价格)买入 100 万英镑。类似地,当在 6 个月后汇率降到 1.2000 时,对企业来讲,远期合约价值为 -23000 美元,这是因为由于持有远期合约而使企业比从市场直接购买英镑多花了 23000 美元。

\section{期权}
期权产品可以分成两种基本类型:看涨期权(call option)的持有者有权在将来某一特定时间以某一特定价格买入某种资产,看跌期权(put option)的持有者有权在将来某一特定时间以某一特定价格卖出某种资产。合约中所说的特定价格叫执行价格(exercise price)或敲定价格(strike price);期权产品所指的特定时间叫到期日(expiration date)或期限(maturity)。美式期权(American option)是指期权持有人在到期日之前任何时间都可以选择行使期权;欧式期权(European option)是指期权持有人只能在到期日才能选择是否行使期权。

期权的买入方被称为持有多头,期权的卖出方被称为持有空头,卖出期权也被称为对期权承约(\textbf{writing the option})。
\section{对冲者}
\textbf{对冲的目的是降低风险,对冲后的实际结果并不一定能保证比不对冲更好。}

采用远期对冲与采用期权对冲有一个关键性区别:\textbf{以远期合约来中和风险的形式是通过设定买入和卖出标的资产的价格来对冲的,而期权产品则提供了价格保险}。当价格向不利方向变化时,期权产品对投资者提供了保护,但同时又能使投资者在价格向有利方向变化时盈利。与远期不同,拥有期权需要付费。
\section{投机者}
对冲者是想避免面对将来资产价格不利波动的风险敞口,而投机者却要建立头寸:他们或者对资产价格上涨下注,或者对资产价格下跌下注。